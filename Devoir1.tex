% Options for packages loaded elsewhere
\PassOptionsToPackage{unicode}{hyperref}
\PassOptionsToPackage{hyphens}{url}
%
\documentclass[
]{article}
\usepackage{amsmath,amssymb}
\usepackage{iftex}
\ifPDFTeX
  \usepackage[T1]{fontenc}
  \usepackage[utf8]{inputenc}
  \usepackage{textcomp} % provide euro and other symbols
\else % if luatex or xetex
  \usepackage{unicode-math} % this also loads fontspec
  \defaultfontfeatures{Scale=MatchLowercase}
  \defaultfontfeatures[\rmfamily]{Ligatures=TeX,Scale=1}
\fi
\usepackage{lmodern}
\ifPDFTeX\else
  % xetex/luatex font selection
\fi
% Use upquote if available, for straight quotes in verbatim environments
\IfFileExists{upquote.sty}{\usepackage{upquote}}{}
\IfFileExists{microtype.sty}{% use microtype if available
  \usepackage[]{microtype}
  \UseMicrotypeSet[protrusion]{basicmath} % disable protrusion for tt fonts
}{}
\makeatletter
\@ifundefined{KOMAClassName}{% if non-KOMA class
  \IfFileExists{parskip.sty}{%
    \usepackage{parskip}
  }{% else
    \setlength{\parindent}{0pt}
    \setlength{\parskip}{6pt plus 2pt minus 1pt}}
}{% if KOMA class
  \KOMAoptions{parskip=half}}
\makeatother
\usepackage{xcolor}
\usepackage[margin=1in]{geometry}
\usepackage{color}
\usepackage{fancyvrb}
\newcommand{\VerbBar}{|}
\newcommand{\VERB}{\Verb[commandchars=\\\{\}]}
\DefineVerbatimEnvironment{Highlighting}{Verbatim}{commandchars=\\\{\}}
% Add ',fontsize=\small' for more characters per line
\usepackage{framed}
\definecolor{shadecolor}{RGB}{248,248,248}
\newenvironment{Shaded}{\begin{snugshade}}{\end{snugshade}}
\newcommand{\AlertTok}[1]{\textcolor[rgb]{0.94,0.16,0.16}{#1}}
\newcommand{\AnnotationTok}[1]{\textcolor[rgb]{0.56,0.35,0.01}{\textbf{\textit{#1}}}}
\newcommand{\AttributeTok}[1]{\textcolor[rgb]{0.13,0.29,0.53}{#1}}
\newcommand{\BaseNTok}[1]{\textcolor[rgb]{0.00,0.00,0.81}{#1}}
\newcommand{\BuiltInTok}[1]{#1}
\newcommand{\CharTok}[1]{\textcolor[rgb]{0.31,0.60,0.02}{#1}}
\newcommand{\CommentTok}[1]{\textcolor[rgb]{0.56,0.35,0.01}{\textit{#1}}}
\newcommand{\CommentVarTok}[1]{\textcolor[rgb]{0.56,0.35,0.01}{\textbf{\textit{#1}}}}
\newcommand{\ConstantTok}[1]{\textcolor[rgb]{0.56,0.35,0.01}{#1}}
\newcommand{\ControlFlowTok}[1]{\textcolor[rgb]{0.13,0.29,0.53}{\textbf{#1}}}
\newcommand{\DataTypeTok}[1]{\textcolor[rgb]{0.13,0.29,0.53}{#1}}
\newcommand{\DecValTok}[1]{\textcolor[rgb]{0.00,0.00,0.81}{#1}}
\newcommand{\DocumentationTok}[1]{\textcolor[rgb]{0.56,0.35,0.01}{\textbf{\textit{#1}}}}
\newcommand{\ErrorTok}[1]{\textcolor[rgb]{0.64,0.00,0.00}{\textbf{#1}}}
\newcommand{\ExtensionTok}[1]{#1}
\newcommand{\FloatTok}[1]{\textcolor[rgb]{0.00,0.00,0.81}{#1}}
\newcommand{\FunctionTok}[1]{\textcolor[rgb]{0.13,0.29,0.53}{\textbf{#1}}}
\newcommand{\ImportTok}[1]{#1}
\newcommand{\InformationTok}[1]{\textcolor[rgb]{0.56,0.35,0.01}{\textbf{\textit{#1}}}}
\newcommand{\KeywordTok}[1]{\textcolor[rgb]{0.13,0.29,0.53}{\textbf{#1}}}
\newcommand{\NormalTok}[1]{#1}
\newcommand{\OperatorTok}[1]{\textcolor[rgb]{0.81,0.36,0.00}{\textbf{#1}}}
\newcommand{\OtherTok}[1]{\textcolor[rgb]{0.56,0.35,0.01}{#1}}
\newcommand{\PreprocessorTok}[1]{\textcolor[rgb]{0.56,0.35,0.01}{\textit{#1}}}
\newcommand{\RegionMarkerTok}[1]{#1}
\newcommand{\SpecialCharTok}[1]{\textcolor[rgb]{0.81,0.36,0.00}{\textbf{#1}}}
\newcommand{\SpecialStringTok}[1]{\textcolor[rgb]{0.31,0.60,0.02}{#1}}
\newcommand{\StringTok}[1]{\textcolor[rgb]{0.31,0.60,0.02}{#1}}
\newcommand{\VariableTok}[1]{\textcolor[rgb]{0.00,0.00,0.00}{#1}}
\newcommand{\VerbatimStringTok}[1]{\textcolor[rgb]{0.31,0.60,0.02}{#1}}
\newcommand{\WarningTok}[1]{\textcolor[rgb]{0.56,0.35,0.01}{\textbf{\textit{#1}}}}
\usepackage{graphicx}
\makeatletter
\def\maxwidth{\ifdim\Gin@nat@width>\linewidth\linewidth\else\Gin@nat@width\fi}
\def\maxheight{\ifdim\Gin@nat@height>\textheight\textheight\else\Gin@nat@height\fi}
\makeatother
% Scale images if necessary, so that they will not overflow the page
% margins by default, and it is still possible to overwrite the defaults
% using explicit options in \includegraphics[width, height, ...]{}
\setkeys{Gin}{width=\maxwidth,height=\maxheight,keepaspectratio}
% Set default figure placement to htbp
\makeatletter
\def\fps@figure{htbp}
\makeatother
\setlength{\emergencystretch}{3em} % prevent overfull lines
\providecommand{\tightlist}{%
  \setlength{\itemsep}{0pt}\setlength{\parskip}{0pt}}
\setcounter{secnumdepth}{-\maxdimen} % remove section numbering
\ifLuaTeX
  \usepackage{selnolig}  % disable illegal ligatures
\fi
\IfFileExists{bookmark.sty}{\usepackage{bookmark}}{\usepackage{hyperref}}
\IfFileExists{xurl.sty}{\usepackage{xurl}}{} % add URL line breaks if available
\urlstyle{same}
\hypersetup{
  pdftitle={Devoir Élections},
  pdfauthor={Gabriel Houdry--Bohême, Adem Bensalem, Milan Soragna, Hector Ménétrier, Serigne Massamba Guèye},
  hidelinks,
  pdfcreator={LaTeX via pandoc}}

\title{Devoir Élections}
\author{Gabriel Houdry--Bohême, Adem Bensalem, Milan Soragna, Hector
Ménétrier, Serigne Massamba Guèye}
\date{}

\begin{document}
\maketitle

\hypertarget{i.-pruxe9sentation-de-la-base-de-donnuxe9es}{%
\subsection{I. Présentation de la base de
données}\label{i.-pruxe9sentation-de-la-base-de-donnuxe9es}}

La base de données choisie est celle des résultats du premier tour des
élections présidentielles françaises de 2022. Elle contient pour chaque
commune française le résultat de votes de chaque candidat, avec les
votes blancs et les abstensions. Afin de l'analyser, chargeons d'abord
les bibliothèques nécessaires.

\begin{Shaded}
\begin{Highlighting}[]
\FunctionTok{library}\NormalTok{(tidyverse)}
\end{Highlighting}
\end{Shaded}

\begin{verbatim}
## -- Attaching core tidyverse packages ------------------------ tidyverse 2.0.0 --
## v dplyr     1.1.4     v readr     2.1.5
## v forcats   1.0.0     v stringr   1.5.1
## v ggplot2   3.4.4     v tibble    3.2.1
## v lubridate 1.9.3     v tidyr     1.3.1
## v purrr     1.0.2     
## -- Conflicts ------------------------------------------ tidyverse_conflicts() --
## x dplyr::filter() masks stats::filter()
## x dplyr::lag()    masks stats::lag()
## i Use the conflicted package (<http://conflicted.r-lib.org/>) to force all conflicts to become errors
\end{verbatim}

\begin{Shaded}
\begin{Highlighting}[]
\FunctionTok{library}\NormalTok{(stringr)}
\FunctionTok{library}\NormalTok{(stringi)}
\FunctionTok{library}\NormalTok{(readxl)}
\FunctionTok{library}\NormalTok{(dplyr)}
\end{Highlighting}
\end{Shaded}

La base de donnée est importée par la commande suivante.

\begin{Shaded}
\begin{Highlighting}[]
\CommentTok{\# Récupération des résultats}
\NormalTok{dataset }\OtherTok{\textless{}{-}} \FunctionTok{read\_xlsx}\NormalTok{(}\StringTok{"dataset\_elections.xlsx"}\NormalTok{)}
\end{Highlighting}
\end{Shaded}

\begin{verbatim}
## New names:
## * `` -> `...29`
## * `` -> `...30`
## * `` -> `...31`
## * `` -> `...32`
## * `` -> `...34`
## * `` -> `...35`
## * `` -> `...36`
## * `` -> `...37`
## * `` -> `...38`
## * `` -> `...39`
## * `` -> `...41`
## * `` -> `...42`
## * `` -> `...43`
## * `` -> `...44`
## * `` -> `...45`
## * `` -> `...46`
## * `` -> `...48`
## * `` -> `...49`
## * `` -> `...50`
## * `` -> `...51`
## * `` -> `...52`
## * `` -> `...53`
## * `` -> `...55`
## * `` -> `...56`
## * `` -> `...57`
## * `` -> `...58`
## * `` -> `...59`
## * `` -> `...60`
## * `` -> `...62`
## * `` -> `...63`
## * `` -> `...64`
## * `` -> `...65`
## * `` -> `...66`
## * `` -> `...67`
## * `` -> `...69`
## * `` -> `...70`
## * `` -> `...71`
## * `` -> `...72`
## * `` -> `...73`
## * `` -> `...74`
## * `` -> `...76`
## * `` -> `...77`
## * `` -> `...78`
## * `` -> `...79`
## * `` -> `...80`
## * `` -> `...81`
## * `` -> `...83`
## * `` -> `...84`
## * `` -> `...85`
## * `` -> `...86`
## * `` -> `...87`
## * `` -> `...88`
## * `` -> `...90`
## * `` -> `...91`
## * `` -> `...92`
## * `` -> `...93`
## * `` -> `...94`
## * `` -> `...95`
## * `` -> `...97`
## * `` -> `...98`
## * `` -> `...99`
## * `` -> `...100`
## * `` -> `...101`
## * `` -> `...102`
## * `` -> `...104`
## * `` -> `...105`
\end{verbatim}

\hypertarget{ii.-analyse-des-votes-par-candidat}{%
\subsection{II. Analyse des votes par
candidat}\label{ii.-analyse-des-votes-par-candidat}}

Ensuite, nous devons effectuer un nettoyage des données. En effet,
toutes les informations ne sont pas utiles pour la visualisation, et
nous souhaitons les mettre sous une forme plus simple à analyser. Ici,
pour une visualisation sous forme de carte, nous regroupons les
résultats par département.

\begin{Shaded}
\begin{Highlighting}[]
\CommentTok{\# Nettoyage des données et aggrégation des résultats}
\NormalTok{dataset\_cleaned }\OtherTok{\textless{}{-}}\NormalTok{ dataset }\SpecialCharTok{\%\textgreater{}\%}
  \FunctionTok{group\_by}\NormalTok{(}\StringTok{\textasciigrave{}}\AttributeTok{Libellé du département}\StringTok{\textasciigrave{}}\NormalTok{) }\SpecialCharTok{\%\textgreater{}\%}
  \FunctionTok{summarise}\NormalTok{(}\AttributeTok{total\_votes =} \FunctionTok{sum}\NormalTok{(Exprimés), }
            \AttributeTok{total\_blancs =} \FunctionTok{sum}\NormalTok{(Blancs),}
            \AttributeTok{pourcentage\_blanc =}\NormalTok{ total\_blancs }\SpecialCharTok{/} \FunctionTok{sum}\NormalTok{(Votants) }\SpecialCharTok{*} \DecValTok{100}\NormalTok{, }
            \AttributeTok{total\_abs =} \FunctionTok{sum}\NormalTok{(}\FunctionTok{as.numeric}\NormalTok{(Abstentions)), }
            \AttributeTok{pourcentage\_abs =}\NormalTok{ total\_abs }\SpecialCharTok{/} \FunctionTok{sum}\NormalTok{(}\FunctionTok{as.numeric}\NormalTok{(Inscrits))}\SpecialCharTok{*} \DecValTok{100}\NormalTok{,}
            \AttributeTok{total\_arthaud =} \FunctionTok{sum}\NormalTok{(VoixArthaud), }
            \AttributeTok{pourcentage\_arthaud =}\NormalTok{ total\_arthaud }\SpecialCharTok{/}\NormalTok{ total\_votes }\SpecialCharTok{*} \DecValTok{100}\NormalTok{,}
            \AttributeTok{total\_roussel =} \FunctionTok{sum}\NormalTok{(VoixRoussel),}
            \AttributeTok{pourcentage\_roussel =}\NormalTok{ total\_roussel }\SpecialCharTok{/}\NormalTok{ total\_votes }\SpecialCharTok{*} \DecValTok{100}\NormalTok{,}
            \AttributeTok{total\_macron =} \FunctionTok{sum}\NormalTok{(VoixMacron),}
            \AttributeTok{pourcentage\_macron =}\NormalTok{ total\_macron }\SpecialCharTok{/}\NormalTok{ total\_votes }\SpecialCharTok{*} \DecValTok{100}\NormalTok{,}
            \AttributeTok{total\_lassalle =} \FunctionTok{sum}\NormalTok{(VoixLassalle),}
            \AttributeTok{pourcentage\_lassalle =}\NormalTok{ total\_lassalle }\SpecialCharTok{/}\NormalTok{ total\_votes }\SpecialCharTok{*} \DecValTok{100}\NormalTok{,}
            \AttributeTok{total\_lepen =} \FunctionTok{sum}\NormalTok{(VoixLePen),}
            \AttributeTok{pourcentage\_lepen =}\NormalTok{ total\_lepen }\SpecialCharTok{/}\NormalTok{ total\_votes }\SpecialCharTok{*} \DecValTok{100}\NormalTok{,}
            \AttributeTok{total\_zemmour =} \FunctionTok{sum}\NormalTok{(VoixZemmour),}
            \AttributeTok{pourcentage\_zemmour =}\NormalTok{ total\_zemmour }\SpecialCharTok{/}\NormalTok{ total\_votes }\SpecialCharTok{*} \DecValTok{100}\NormalTok{,}
            \AttributeTok{total\_melenchon =} \FunctionTok{sum}\NormalTok{(VoixMelenchon),}
            \AttributeTok{pourcentage\_melenchon =}\NormalTok{ total\_melenchon }\SpecialCharTok{/}\NormalTok{ total\_votes }\SpecialCharTok{*} \DecValTok{100}\NormalTok{,}
            \AttributeTok{total\_hidalgo =} \FunctionTok{sum}\NormalTok{(VoixHidalgo),}
            \AttributeTok{pourcentage\_hidalgo =}\NormalTok{ total\_hidalgo }\SpecialCharTok{/}\NormalTok{ total\_votes }\SpecialCharTok{*} \DecValTok{100}\NormalTok{,}
            \AttributeTok{total\_jadot =} \FunctionTok{sum}\NormalTok{(VoixJadot),}
            \AttributeTok{pourcentage\_jadot =}\NormalTok{ total\_jadot }\SpecialCharTok{/}\NormalTok{ total\_votes }\SpecialCharTok{*} \DecValTok{100}\NormalTok{,}
            \AttributeTok{total\_pecresse =} \FunctionTok{sum}\NormalTok{(VoixPecresse),}
            \AttributeTok{pourcentage\_pecresse =}\NormalTok{ total\_pecresse }\SpecialCharTok{/}\NormalTok{ total\_votes }\SpecialCharTok{*} \DecValTok{100}\NormalTok{,}
            \AttributeTok{total\_poutou =} \FunctionTok{sum}\NormalTok{(VoixPoutou),}
            \AttributeTok{pourcentage\_poutou =}\NormalTok{ total\_poutou }\SpecialCharTok{/}\NormalTok{ total\_votes }\SpecialCharTok{*} \DecValTok{100}\NormalTok{,}
            \AttributeTok{total\_dupontaignan =} \FunctionTok{sum}\NormalTok{(VoixDupontAignan),}
            \AttributeTok{pourcentage\_dupontaignan =}\NormalTok{ total\_dupontaignan }\SpecialCharTok{/}\NormalTok{ total\_votes }\SpecialCharTok{*} \DecValTok{100}\NormalTok{)}
\end{Highlighting}
\end{Shaded}

Nous pouvons désormais observer la moyenne de votes par département pour
chaque candidat.

\begin{Shaded}
\begin{Highlighting}[]
\CommentTok{\#Moyenne de vote de chaque candidat par département}
\NormalTok{moy\_votes }\OtherTok{=} \FunctionTok{mean}\NormalTok{(dataset\_cleaned}\SpecialCharTok{$}\NormalTok{total\_votes)}
\NormalTok{moy\_abs }\OtherTok{=} \FunctionTok{mean}\NormalTok{(dataset\_cleaned}\SpecialCharTok{$}\NormalTok{total\_abs)}
\NormalTok{moy\_blanc }\OtherTok{=} \FunctionTok{mean}\NormalTok{(dataset\_cleaned}\SpecialCharTok{$}\NormalTok{total\_blancs)}
\NormalTok{moy\_votes\_candidat }\OtherTok{\textless{}{-}} \FunctionTok{c}\NormalTok{(}\StringTok{"N. Arthaud"} \OtherTok{=} \FunctionTok{mean}\NormalTok{(dataset\_cleaned}\SpecialCharTok{$}\NormalTok{total\_arthaud), }
                       \StringTok{"P. Poutou"} \OtherTok{=} \FunctionTok{mean}\NormalTok{(dataset\_cleaned}\SpecialCharTok{$}\NormalTok{total\_poutou),}
                       \StringTok{"F. Roussel"} \OtherTok{=} \FunctionTok{mean}\NormalTok{(dataset\_cleaned}\SpecialCharTok{$}\NormalTok{total\_roussel),}
                       \StringTok{"J{-}L. Mélenchon"} \OtherTok{=} \FunctionTok{mean}\NormalTok{(dataset\_cleaned}\SpecialCharTok{$}\NormalTok{total\_melenchon),}
                       \StringTok{"A. Hidalgo"} \OtherTok{=} \FunctionTok{mean}\NormalTok{(dataset\_cleaned}\SpecialCharTok{$}\NormalTok{total\_hidalgo),}
                       \StringTok{"Y. Jadot"} \OtherTok{=} \FunctionTok{mean}\NormalTok{(dataset\_cleaned}\SpecialCharTok{$}\NormalTok{total\_jadot),}
                       \StringTok{"E. Macron"} \OtherTok{=} \FunctionTok{mean}\NormalTok{(dataset\_cleaned}\SpecialCharTok{$}\NormalTok{total\_macron), }
                       \StringTok{"J. Lassalle"} \OtherTok{=} \FunctionTok{mean}\NormalTok{(dataset\_cleaned}\SpecialCharTok{$}\NormalTok{total\_lassalle),}
                       \StringTok{"V. Pecresse"} \OtherTok{=} \FunctionTok{mean}\NormalTok{(dataset\_cleaned}\SpecialCharTok{$}\NormalTok{total\_pecresse),}
                       \StringTok{"M. Le Pen"} \OtherTok{=} \FunctionTok{mean}\NormalTok{(dataset\_cleaned}\SpecialCharTok{$}\NormalTok{total\_lepen),}
                       \StringTok{"N. Dupont{-}Aignan"} \OtherTok{=} \FunctionTok{mean}\NormalTok{(dataset\_cleaned}\SpecialCharTok{$}\NormalTok{total\_dupontaignan),}
                       \StringTok{"É. Zemmour"} \OtherTok{=} \FunctionTok{mean}\NormalTok{(dataset\_cleaned}\SpecialCharTok{$}\NormalTok{total\_zemmour))}
\CommentTok{\#Tri des candidats par moyenne de votes.}
\NormalTok{sorted\_avg\_votes }\OtherTok{\textless{}{-}} \FunctionTok{sort}\NormalTok{(moy\_votes\_candidat, }\AttributeTok{decreasing =} \ConstantTok{TRUE}\NormalTok{)}
\FunctionTok{options}\NormalTok{(}\AttributeTok{scipen =} \DecValTok{999}\NormalTok{)}
\CommentTok{\# Création du graphe}
\FunctionTok{barplot}\NormalTok{(sorted\_avg\_votes, }
        \AttributeTok{main =} \StringTok{"Moyenne de votes par candidat par département"}\NormalTok{,}
        \AttributeTok{xlab =} \StringTok{"Candidats"}\NormalTok{,}
        \AttributeTok{ylab =} \StringTok{"Moyenne de votes par département"}\NormalTok{,}
        \AttributeTok{col =} \StringTok{"skyblue"}\NormalTok{,}
        \AttributeTok{ylim =} \FunctionTok{c}\NormalTok{(}\DecValTok{0}\NormalTok{, }\FunctionTok{max}\NormalTok{(sorted\_avg\_votes) }\SpecialCharTok{*} \FloatTok{1.1}\NormalTok{),}
        \AttributeTok{las =} \DecValTok{2}\NormalTok{)}
\end{Highlighting}
\end{Shaded}

\begin{center}\includegraphics{Devoir1_files/figure-latex/moyenne-1} \end{center}

On peut également représenter les pourcentages d'obtention de vote des
candidats grâce à un diagramme circulaire.

\begin{Shaded}
\begin{Highlighting}[]
\CommentTok{\# Couleur pour les candidats}
\NormalTok{candidate\_colors }\OtherTok{\textless{}{-}} \FunctionTok{c}\NormalTok{(}\StringTok{"N. Arthaud"} \OtherTok{=} \StringTok{"red4"}\NormalTok{, }
                      \StringTok{"P. Poutou"} \OtherTok{=} \StringTok{"red3"}\NormalTok{,}
                      \StringTok{"F. Roussel"} \OtherTok{=} \StringTok{"red2"}\NormalTok{,}
                      \StringTok{"J{-}L. Mélenchon"} \OtherTok{=} \StringTok{"tomato2"}\NormalTok{,}
                      \StringTok{"A. Hidalgo"} \OtherTok{=} \StringTok{"salmon2"}\NormalTok{,}
                      \StringTok{"Y. Jadot"} \OtherTok{=} \StringTok{"springgreen4"}\NormalTok{,}
                      \StringTok{"E. Macron"} \OtherTok{=} \StringTok{"goldenrod1"}\NormalTok{, }
                      \StringTok{"J. Lassalle"} \OtherTok{=} \StringTok{"lightblue"}\NormalTok{,}
                      \StringTok{"V. Pecresse"} \OtherTok{=} \StringTok{"royalblue1"}\NormalTok{,}
                      \StringTok{"M. Le Pen"} \OtherTok{=} \StringTok{"blue2"}\NormalTok{,}
                      \StringTok{"N. Dupont{-}Aignan"} \OtherTok{=} \StringTok{"blue3"}\NormalTok{,}
                      \StringTok{"É. Zemmour"} \OtherTok{=} \StringTok{"navyblue"}\NormalTok{)}

\CommentTok{\#pourcentage}
\NormalTok{total\_votes }\OtherTok{=} \FunctionTok{sum}\NormalTok{(dataset\_cleaned}\SpecialCharTok{$}\NormalTok{total\_votes)}
\NormalTok{candidate\_pourcentage }\OtherTok{\textless{}{-}} \FunctionTok{c}\NormalTok{(}\StringTok{"N. Arthaud"} \OtherTok{=} \FunctionTok{sum}\NormalTok{(dataset\_cleaned}\SpecialCharTok{$}\NormalTok{total\_arthaud)}\SpecialCharTok{/}\NormalTok{total\_votes}\SpecialCharTok{*}\DecValTok{100}\NormalTok{,}
                           \StringTok{"P. Poutou"} \OtherTok{=} \FunctionTok{sum}\NormalTok{(dataset\_cleaned}\SpecialCharTok{$}\NormalTok{total\_poutou)}\SpecialCharTok{/}\NormalTok{total\_votes}\SpecialCharTok{*}\DecValTok{100}\NormalTok{,}
                           \StringTok{"F. Roussel"} \OtherTok{=} \FunctionTok{sum}\NormalTok{(dataset\_cleaned}\SpecialCharTok{$}\NormalTok{total\_roussel)}\SpecialCharTok{/}\NormalTok{total\_votes}\SpecialCharTok{*}\DecValTok{100}\NormalTok{,}
                           \StringTok{"J{-}L. Mélenchon"} \OtherTok{=} \FunctionTok{sum}\NormalTok{(dataset\_cleaned}\SpecialCharTok{$}\NormalTok{total\_melenchon)}\SpecialCharTok{/}\NormalTok{total\_votes}\SpecialCharTok{*}\DecValTok{100}\NormalTok{,}
                           \StringTok{"A. Hidalgo"} \OtherTok{=} \FunctionTok{sum}\NormalTok{(dataset\_cleaned}\SpecialCharTok{$}\NormalTok{total\_hidalgo)}\SpecialCharTok{/}\NormalTok{total\_votes}\SpecialCharTok{*}\DecValTok{100}\NormalTok{,}
                           \StringTok{"Y. Jadot"} \OtherTok{=} \FunctionTok{sum}\NormalTok{(dataset\_cleaned}\SpecialCharTok{$}\NormalTok{total\_jadot)}\SpecialCharTok{/}\NormalTok{total\_votes}\SpecialCharTok{*}\DecValTok{100}\NormalTok{,}
                           \StringTok{"E. Macron"} \OtherTok{=} \FunctionTok{sum}\NormalTok{(dataset\_cleaned}\SpecialCharTok{$}\NormalTok{total\_macron)}\SpecialCharTok{/}\NormalTok{total\_votes}\SpecialCharTok{*}\DecValTok{100}\NormalTok{,}
                           \StringTok{"J. Lassalle"} \OtherTok{=} \FunctionTok{sum}\NormalTok{(dataset\_cleaned}\SpecialCharTok{$}\NormalTok{total\_lassalle)}\SpecialCharTok{/}\NormalTok{total\_votes}\SpecialCharTok{*}\DecValTok{100}\NormalTok{,}
                           \StringTok{"V. Pecresse"} \OtherTok{=} \FunctionTok{sum}\NormalTok{(dataset\_cleaned}\SpecialCharTok{$}\NormalTok{total\_pecresse)}\SpecialCharTok{/}\NormalTok{total\_votes}\SpecialCharTok{*}\DecValTok{100}\NormalTok{,}
                           \StringTok{"M. Le Pen"} \OtherTok{=} \FunctionTok{sum}\NormalTok{(dataset\_cleaned}\SpecialCharTok{$}\NormalTok{total\_lepen)}\SpecialCharTok{/}\NormalTok{total\_votes}\SpecialCharTok{*}\DecValTok{100}\NormalTok{,}
                           \StringTok{"N. Dupont{-}Aignan"} \OtherTok{=} \FunctionTok{sum}\NormalTok{(dataset\_cleaned}\SpecialCharTok{$}\NormalTok{total\_dupontaignan)}\SpecialCharTok{/}\NormalTok{total\_votes}\SpecialCharTok{*}\DecValTok{100}\NormalTok{,}
                           \StringTok{"É. Zemmour"} \OtherTok{=} \FunctionTok{sum}\NormalTok{(dataset\_cleaned}\SpecialCharTok{$}\NormalTok{total\_zemmour)}\SpecialCharTok{/}\NormalTok{total\_votes}\SpecialCharTok{*}\DecValTok{100}\NormalTok{)}

\CommentTok{\# Données}
\NormalTok{noms\_candidats }\OtherTok{\textless{}{-}} \FunctionTok{names}\NormalTok{(sorted\_avg\_votes)}
\NormalTok{moyennes\_votes }\OtherTok{\textless{}{-}} \FunctionTok{unname}\NormalTok{(sorted\_avg\_votes)}


\CommentTok{\# Création du diagramme circulaire}
\FunctionTok{par}\NormalTok{(}\AttributeTok{pty =} \StringTok{"m"}\NormalTok{, }\AttributeTok{mfrow =} \FunctionTok{c}\NormalTok{(}\DecValTok{1}\NormalTok{, }\DecValTok{1}\NormalTok{), }\AttributeTok{mar =} \FunctionTok{c}\NormalTok{(}\DecValTok{2}\NormalTok{, }\DecValTok{2}\NormalTok{, }\DecValTok{2}\NormalTok{, }\DecValTok{2}\NormalTok{))}

\FunctionTok{plot.new}\NormalTok{()}

\FunctionTok{pie}\NormalTok{(moyennes\_votes, }
    \AttributeTok{labels =} \FunctionTok{paste}\NormalTok{(}\FunctionTok{round}\NormalTok{(candidate\_pourcentage[noms\_candidats]),}\StringTok{"\%"}\NormalTok{),}
    \AttributeTok{main =} \StringTok{"Pourcentage des votes par candidat"}\NormalTok{, }
    \AttributeTok{col =}\NormalTok{ candidate\_colors[noms\_candidats],}
    \AttributeTok{cex =} \FloatTok{0.6}\NormalTok{)}

\FunctionTok{legend}\NormalTok{(}\StringTok{"right"}\NormalTok{, }
       \AttributeTok{legend =}\NormalTok{ noms\_candidats, }
       \AttributeTok{fill =}\NormalTok{ candidate\_colors[noms\_candidats], }
       \AttributeTok{title =} \StringTok{"Candidats"}\NormalTok{, }
       \AttributeTok{cex =} \FloatTok{0.6}\NormalTok{)}
\end{Highlighting}
\end{Shaded}

\includegraphics{Devoir1_files/figure-latex/unnamed-chunk-4-1.pdf}

Nous pouvons également regarder l'étendue des données que nous venons
d'obtenir.

\begin{Shaded}
\begin{Highlighting}[]
\NormalTok{etendue }\OtherTok{=} \FunctionTok{max}\NormalTok{(moy\_votes\_candidat) }\SpecialCharTok{{-}} \FunctionTok{min}\NormalTok{(moy\_votes\_candidat)}
\FunctionTok{print}\NormalTok{(etendue)}
\end{Highlighting}
\end{Shaded}

\begin{verbatim}
## [1] 89588.45
\end{verbatim}

En moyenne par région, il y a environ 328 345 votants, 119 852
abstentions et 5080 vote blancs. L'étendue des votes entre les candidats
est d'environ 89 588 votes.

\hypertarget{iii.-cartographie-des-ruxe9sultats}{%
\subsection{III. Cartographie des
résultats}\label{iii.-cartographie-des-ruxe9sultats}}

Tout d'abord, nous allons nous intéresser principalement à la France
métropolitaine. Pour ce faire, il suffit de filtrer les départements.

\begin{Shaded}
\begin{Highlighting}[]
\CommentTok{\# Changement de noms de départements}
\FunctionTok{names}\NormalTok{(dataset\_cleaned)[}\DecValTok{1}\NormalTok{] }\OtherTok{\textless{}{-}} \StringTok{"region"}
\NormalTok{dataset\_cleaned}\SpecialCharTok{$}\NormalTok{region }\OtherTok{\textless{}{-}} \FunctionTok{stri\_trans\_general}\NormalTok{(dataset\_cleaned}\SpecialCharTok{$}\NormalTok{region, }\StringTok{"Latin{-}ASCII"}\NormalTok{) }\SpecialCharTok{\%\textgreater{}\%}
  \FunctionTok{str\_replace\_all}\NormalTok{(}\StringTok{"Cote{-}d\textquotesingle{}Or"}\NormalTok{, }\StringTok{"Cote{-}Dor"}\NormalTok{) }\SpecialCharTok{\%\textgreater{}\%}
  \FunctionTok{str\_replace\_all}\NormalTok{(}\StringTok{"Cotes{-}d\textquotesingle{}Armor"}\NormalTok{, }\StringTok{"Cotes{-}Darmor"}\NormalTok{) }\SpecialCharTok{\%\textgreater{}\%}
  \FunctionTok{str\_replace\_all}\NormalTok{(}\StringTok{"Corse{-}du{-}Sud"}\NormalTok{, }\StringTok{"Corse du Sud"}\NormalTok{) }\SpecialCharTok{\%\textgreater{}\%}
  \FunctionTok{str\_replace\_all}\NormalTok{(}\StringTok{"Val{-}d\textquotesingle{}Oise"}\NormalTok{, }\StringTok{"Val{-}Doise"}\NormalTok{) }\SpecialCharTok{\%\textgreater{}\%}
  \FunctionTok{str\_replace\_all}\NormalTok{(}\StringTok{"Corse{-}du{-}Sud"}\NormalTok{, }\StringTok{"Corse du Sud"}\NormalTok{)}

\CommentTok{\# Récupération de la carte de France}
\NormalTok{map }\OtherTok{\textless{}{-}} \FunctionTok{map\_data}\NormalTok{(}\StringTok{"france"}\NormalTok{)}

\CommentTok{\# Filtrage des départements pour ne garder que la France métropolitaine}
\NormalTok{dataset\_cleaned\_filtered }\OtherTok{\textless{}{-}}\NormalTok{ dataset\_cleaned }\SpecialCharTok{\%\textgreater{}\%}
  \FunctionTok{filter}\NormalTok{(region }\SpecialCharTok{\%in\%}\NormalTok{ map}\SpecialCharTok{$}\NormalTok{region)}

\CommentTok{\# Fusion avec les données de la carte}
\NormalTok{result\_map }\OtherTok{\textless{}{-}} \FunctionTok{left\_join}\NormalTok{(}\AttributeTok{x =}\NormalTok{ map[,}\SpecialCharTok{{-}}\DecValTok{6}\NormalTok{], }\AttributeTok{y =}\NormalTok{ dataset\_cleaned\_filtered)}
\end{Highlighting}
\end{Shaded}

\begin{verbatim}
## Joining with `by = join_by(region)`
\end{verbatim}

Ensuite, il faut trouver le candidat arrivé en tête dans chaque
département

\begin{Shaded}
\begin{Highlighting}[]
\CommentTok{\# On cherche ici le candidat en tête dans chaque département}
\NormalTok{result\_map }\OtherTok{\textless{}{-}}\NormalTok{ result\_map }\SpecialCharTok{\%\textgreater{}\%}
  \FunctionTok{rowwise}\NormalTok{() }\SpecialCharTok{\%\textgreater{}\%}
  \FunctionTok{mutate}\NormalTok{(}\AttributeTok{candidat\_gagnant =} \FunctionTok{case\_when}\NormalTok{(}
\NormalTok{    total\_arthaud }\SpecialCharTok{==} \FunctionTok{max}\NormalTok{(total\_arthaud, total\_roussel, total\_macron, }
\NormalTok{                         total\_lassalle, total\_lepen, total\_zemmour, }
\NormalTok{                         total\_melenchon, total\_hidalgo, total\_jadot, }
\NormalTok{                         total\_pecresse, total\_poutou, total\_dupontaignan) }\SpecialCharTok{\textasciitilde{}} \StringTok{"N. Arthaud"}\NormalTok{,}
\NormalTok{    total\_roussel }\SpecialCharTok{==} \FunctionTok{max}\NormalTok{(total\_arthaud, total\_roussel, total\_macron, }
\NormalTok{                         total\_lassalle, total\_lepen, total\_zemmour, }
\NormalTok{                         total\_melenchon, total\_hidalgo, total\_jadot, }
\NormalTok{                         total\_pecresse, total\_poutou, total\_dupontaignan) }\SpecialCharTok{\textasciitilde{}} \StringTok{"F. Roussel"}\NormalTok{,}
\NormalTok{    total\_macron }\SpecialCharTok{==} \FunctionTok{max}\NormalTok{(total\_arthaud, total\_roussel, total\_macron, }
\NormalTok{                        total\_lassalle, total\_lepen, total\_zemmour, }
\NormalTok{                        total\_melenchon, total\_hidalgo, total\_jadot, }
\NormalTok{                        total\_pecresse, total\_poutou, total\_dupontaignan) }\SpecialCharTok{\textasciitilde{}} \StringTok{"E. Macron"}\NormalTok{,}
\NormalTok{    total\_lassalle }\SpecialCharTok{==} \FunctionTok{max}\NormalTok{(total\_arthaud, total\_roussel, total\_macron, }
\NormalTok{                          total\_lassalle, total\_lepen, total\_zemmour, }
\NormalTok{                          total\_melenchon, total\_hidalgo, total\_jadot, }
\NormalTok{                          total\_pecresse, total\_poutou, total\_dupontaignan) }\SpecialCharTok{\textasciitilde{}} \StringTok{"J. Lassalle"}\NormalTok{,}
\NormalTok{    total\_lepen }\SpecialCharTok{==} \FunctionTok{max}\NormalTok{(total\_arthaud, total\_roussel, total\_macron, }
\NormalTok{                       total\_lassalle, total\_lepen, total\_zemmour, }
\NormalTok{                       total\_melenchon, total\_hidalgo, total\_jadot, }
\NormalTok{                       total\_pecresse, total\_poutou, total\_dupontaignan) }\SpecialCharTok{\textasciitilde{}} \StringTok{"M. Le Pen"}\NormalTok{,}
\NormalTok{    total\_zemmour }\SpecialCharTok{==} \FunctionTok{max}\NormalTok{(total\_arthaud, total\_roussel, total\_macron, }
\NormalTok{                         total\_lassalle, total\_lepen, total\_zemmour, }
\NormalTok{                         total\_melenchon, total\_hidalgo, total\_jadot, }
\NormalTok{                         total\_pecresse, total\_poutou, total\_dupontaignan) }\SpecialCharTok{\textasciitilde{}} \StringTok{"É. Zemmour"}\NormalTok{,}
\NormalTok{    total\_melenchon }\SpecialCharTok{==} \FunctionTok{max}\NormalTok{(total\_arthaud, total\_roussel, total\_macron, }
\NormalTok{                           total\_lassalle, total\_lepen, total\_zemmour, }
\NormalTok{                           total\_melenchon, total\_hidalgo, total\_jadot, }
\NormalTok{                           total\_pecresse, total\_poutou, total\_dupontaignan) }\SpecialCharTok{\textasciitilde{}} \StringTok{"J{-}L. Mélenchon"}\NormalTok{,}
\NormalTok{    total\_hidalgo }\SpecialCharTok{==} \FunctionTok{max}\NormalTok{(total\_arthaud, total\_roussel, total\_macron, }
\NormalTok{                         total\_lassalle, total\_lepen, total\_zemmour, }
\NormalTok{                         total\_melenchon, total\_hidalgo, total\_jadot, }
\NormalTok{                         total\_pecresse, total\_poutou, total\_dupontaignan) }\SpecialCharTok{\textasciitilde{}} \StringTok{"A. Hidalgo"}\NormalTok{,}
\NormalTok{    total\_jadot }\SpecialCharTok{==} \FunctionTok{max}\NormalTok{(total\_arthaud, total\_roussel, total\_macron, }
\NormalTok{                       total\_lassalle, total\_lepen, total\_zemmour, }
\NormalTok{                       total\_melenchon, total\_hidalgo, total\_jadot, }
\NormalTok{                       total\_pecresse, total\_poutou, total\_dupontaignan) }\SpecialCharTok{\textasciitilde{}} \StringTok{"Y. Jadot"}\NormalTok{,}
\NormalTok{    total\_pecresse }\SpecialCharTok{==} \FunctionTok{max}\NormalTok{(total\_arthaud, total\_roussel, total\_macron, }
\NormalTok{                          total\_lassalle, total\_lepen, total\_zemmour, }
\NormalTok{                          total\_melenchon, total\_hidalgo, total\_jadot, }
\NormalTok{                          total\_pecresse, total\_poutou, total\_dupontaignan) }\SpecialCharTok{\textasciitilde{}} \StringTok{"V. Pécresse"}\NormalTok{,}
\NormalTok{    total\_poutou }\SpecialCharTok{==} \FunctionTok{max}\NormalTok{(total\_arthaud, total\_roussel, total\_macron, }
\NormalTok{                        total\_lassalle, total\_lepen, total\_zemmour, }
\NormalTok{                        total\_melenchon, total\_hidalgo, total\_jadot, }
\NormalTok{                        total\_pecresse, total\_poutou, total\_dupontaignan) }\SpecialCharTok{\textasciitilde{}} \StringTok{"P. Poutou"}\NormalTok{,}
\NormalTok{    total\_dupontaignan }\SpecialCharTok{==} \FunctionTok{max}\NormalTok{(total\_arthaud, total\_roussel, total\_macron, }
\NormalTok{                              total\_lassalle, total\_lepen, total\_zemmour, }
\NormalTok{                              total\_melenchon, total\_hidalgo, total\_jadot, }
\NormalTok{                              total\_pecresse, total\_poutou, total\_dupontaignan) }\SpecialCharTok{\textasciitilde{}} \StringTok{"N. Dupont{-}Aignan"}\NormalTok{,}
    \ConstantTok{TRUE} \SpecialCharTok{\textasciitilde{}} \ConstantTok{NA\_character\_}
\NormalTok{  ))}
\end{Highlighting}
\end{Shaded}

Nous pouvons désormais générer la carte. D'abord, nous assignons une
couleur à chaque candidat. Ensuite, nous créons le thème de la carte qui
sera générée à l'aide de ggplot.

\begin{Shaded}
\begin{Highlighting}[]
\CommentTok{\# Couleur pour les candidats}
\NormalTok{candidate\_colors }\OtherTok{\textless{}{-}} \FunctionTok{c}\NormalTok{(}\StringTok{"N. Arthaud"} \OtherTok{=} \StringTok{"red4"}\NormalTok{, }
                      \StringTok{"P. Poutou"} \OtherTok{=} \StringTok{"red3"}\NormalTok{,}
                      \StringTok{"F. Roussel"} \OtherTok{=} \StringTok{"red2"}\NormalTok{,}
                      \StringTok{"J{-}L. Mélenchon"} \OtherTok{=} \StringTok{"tomato2"}\NormalTok{,}
                      \StringTok{"A. Hidalgo"} \OtherTok{=} \StringTok{"salmon2"}\NormalTok{,}
                      \StringTok{"Y. Jadot"} \OtherTok{=} \StringTok{"springgreen4"}\NormalTok{,}
                      \StringTok{"E. Macron"} \OtherTok{=} \StringTok{"goldenrod1"}\NormalTok{, }
                      \StringTok{"J. Lassalle"} \OtherTok{=} \StringTok{"lightblue"}\NormalTok{,}
                      \StringTok{"V. Pecresse"} \OtherTok{=} \StringTok{"royalblue1"}\NormalTok{,}
                      \StringTok{"M. Le Pen"} \OtherTok{=} \StringTok{"blue2"}\NormalTok{,}
                      \StringTok{"N. Dupont{-}Aignan"} \OtherTok{=} \StringTok{"blue3"}\NormalTok{,}
                      \StringTok{"É. Zemmour"} \OtherTok{=} \StringTok{"navyblue"}\NormalTok{)}

\CommentTok{\# Apparence de la carte}
\NormalTok{map\_theme }\OtherTok{\textless{}{-}} \FunctionTok{theme}\NormalTok{(}\AttributeTok{title =} \FunctionTok{element\_text}\NormalTok{(}\AttributeTok{margin =} \FunctionTok{margin}\NormalTok{(}\AttributeTok{b =} \DecValTok{20}\NormalTok{, }\AttributeTok{r =} \DecValTok{25}\NormalTok{)), }\CommentTok{\# Ajustement des marges intérieures du titre}
                   \AttributeTok{plot.title =} \FunctionTok{element\_text}\NormalTok{(}\AttributeTok{margin =} \FunctionTok{margin}\NormalTok{(}\AttributeTok{b =} \DecValTok{20}\NormalTok{, }\AttributeTok{t =} \DecValTok{20}\NormalTok{)), }\CommentTok{\# Ajustement des marges intérieures du titre principal}
                   \AttributeTok{plot.subtitle =} \FunctionTok{element\_text}\NormalTok{(}\AttributeTok{margin =} \FunctionTok{margin}\NormalTok{(}\AttributeTok{b =} \DecValTok{20}\NormalTok{)), }\CommentTok{\# Ajustement des marges intérieures du sous{-}titre}
                   \AttributeTok{axis.text.x=}\FunctionTok{element\_blank}\NormalTok{(),}
                   \AttributeTok{axis.text.y=}\FunctionTok{element\_blank}\NormalTok{(),}
                   \AttributeTok{axis.ticks=}\FunctionTok{element\_blank}\NormalTok{(),}
                   \AttributeTok{axis.title.x=}\FunctionTok{element\_blank}\NormalTok{(),}
                   \AttributeTok{axis.title.y=}\FunctionTok{element\_blank}\NormalTok{(),}
                   \AttributeTok{panel.grid.major=} \FunctionTok{element\_blank}\NormalTok{(), }
                   \AttributeTok{panel.background=} \FunctionTok{element\_blank}\NormalTok{()) }

\CommentTok{\# Création de la carte}
\FunctionTok{ggplot}\NormalTok{(result\_map, }\FunctionTok{aes}\NormalTok{(long, lat, }\AttributeTok{group =}\NormalTok{ group, }\AttributeTok{fill =}\NormalTok{ candidat\_gagnant)) }\SpecialCharTok{+}
  \FunctionTok{geom\_polygon}\NormalTok{() }\SpecialCharTok{+}
  \FunctionTok{geom\_path}\NormalTok{(}\AttributeTok{color =} \StringTok{"white"}\NormalTok{, }\AttributeTok{size =} \FloatTok{0.3}\NormalTok{) }\SpecialCharTok{+} \CommentTok{\# On ajoute la frontière entre les départements}
  \FunctionTok{coord\_map}\NormalTok{() }\SpecialCharTok{+}
  \FunctionTok{scale\_fill\_manual}\NormalTok{(}\AttributeTok{values =}\NormalTok{ candidate\_colors, }\AttributeTok{name =} \StringTok{"Candidat arrivé en tête"}\NormalTok{) }\SpecialCharTok{+}
  \FunctionTok{labs}\NormalTok{(}\AttributeTok{x =} \StringTok{""}\NormalTok{, }
       \AttributeTok{y =} \StringTok{""}\NormalTok{, }
       \AttributeTok{title =} \StringTok{"Candidat arrivé en tête par département au premier tour des présidentielles 2022"}\NormalTok{, }
       \AttributeTok{subtitle =} \StringTok{"Données via data.gouv"}\NormalTok{) }\SpecialCharTok{+}
\NormalTok{  map\_theme}
\end{Highlighting}
\end{Shaded}

\begin{verbatim}
## Warning: Using `size` aesthetic for lines was deprecated in ggplot2 3.4.0.
## i Please use `linewidth` instead.
## This warning is displayed once every 8 hours.
## Call `lifecycle::last_lifecycle_warnings()` to see where this warning was
## generated.
\end{verbatim}

\begin{center}\includegraphics{Devoir1_files/figure-latex/carte-1} \end{center}

\hypertarget{iv.-ruxe9gression-linuxe9aire}{%
\subsection{IV. Régression
linéaire}\label{iv.-ruxe9gression-linuxe9aire}}

\begin{Shaded}
\begin{Highlighting}[]
\NormalTok{graph\_reg\_lin }\OtherTok{\textless{}{-}} \ControlFlowTok{function}\NormalTok{(x, y , x\_lab, y\_lab,main\_lab) \{}
\NormalTok{  modele\_regression }\OtherTok{\textless{}{-}} \FunctionTok{lm}\NormalTok{(y }\SpecialCharTok{\textasciitilde{}}\NormalTok{ x)}
  
  \FunctionTok{plot}\NormalTok{(x, y, }
       \AttributeTok{xlab =}\NormalTok{ x\_lab, }
       \AttributeTok{ylab =}\NormalTok{ y\_lab, }
       \AttributeTok{main =}\NormalTok{ main\_lab,}
       \AttributeTok{pch =} \DecValTok{19}\NormalTok{, }\AttributeTok{col =} \StringTok{"blue"}\NormalTok{)}
  
  \FunctionTok{abline}\NormalTok{(modele\_regression, }\AttributeTok{col =} \StringTok{"red"}\NormalTok{)}
  
  \FunctionTok{legend}\NormalTok{(}\StringTok{"topleft"}\NormalTok{, }\AttributeTok{legend =} \FunctionTok{c}\NormalTok{(}\StringTok{"Données"}\NormalTok{, }\StringTok{"Ligne de régression"}\NormalTok{),}
         \AttributeTok{col =} \FunctionTok{c}\NormalTok{(}\StringTok{"blue"}\NormalTok{, }\StringTok{"red"}\NormalTok{), }\AttributeTok{pch =} \FunctionTok{c}\NormalTok{(}\DecValTok{19}\NormalTok{, }\ConstantTok{NA}\NormalTok{), }\AttributeTok{lty =} \FunctionTok{c}\NormalTok{(}\ConstantTok{NA}\NormalTok{, }\DecValTok{1}\NormalTok{),}
         \AttributeTok{title =} \StringTok{"Légende"}\NormalTok{, }\AttributeTok{cex =} \FloatTok{0.3}\NormalTok{)}
\NormalTok{\}}
\end{Highlighting}
\end{Shaded}

Macron

\begin{Shaded}
\begin{Highlighting}[]
\FunctionTok{graph\_reg\_lin}\NormalTok{(dataset\_cleaned}\SpecialCharTok{$}\NormalTok{total\_votes, dataset\_cleaned}\SpecialCharTok{$}\NormalTok{total\_macron, }\StringTok{"Total des votes par département"}\NormalTok{,}\StringTok{"Total des votes pour Macron par département"}\NormalTok{,}\StringTok{"Relation entre les votes pour Macron et le total des votes"}\NormalTok{)}
\end{Highlighting}
\end{Shaded}

\includegraphics{Devoir1_files/figure-latex/unnamed-chunk-9-1.pdf}

Arthaud

\begin{Shaded}
\begin{Highlighting}[]
\FunctionTok{graph\_reg\_lin}\NormalTok{(dataset\_cleaned}\SpecialCharTok{$}\NormalTok{total\_votes, dataset\_cleaned}\SpecialCharTok{$}\NormalTok{total\_arthaud, }\StringTok{"Total des votes par département"}\NormalTok{,}\StringTok{"Total des votes pour Arthaud par département"}\NormalTok{,}\StringTok{"Relation entre les votes pour Arthaud et le total des votes"}\NormalTok{)}
\end{Highlighting}
\end{Shaded}

\includegraphics{Devoir1_files/figure-latex/unnamed-chunk-10-1.pdf}

Si on compare ces deux graphiques de régression linéaire, on peut en
conclure que plus il y aura de votes, plus les votes des candidats
augmenteront mais cette évolution des votes sera plus ou moins rapide
selon les candidats (ici les votes d'E. Macron augmenteront beaucoup
plus rapidement que ceux de N. Arthaud).

\end{document}
